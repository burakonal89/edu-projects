\chapter{CONCLUSION}
\label{chp:conclusion}
In this thesis, we suggested and tested methods, some of which were already suggested, to estimate the curve and phase of the airflow over the mouth by using the sounds recorded on chest wall. The airflow and its phase have diagnostic value, so the accurate estimation of them may increase pulmonary disease detector's performance in case there is no airflow measurement.
\par In chapter \ref{chp:airflow_curve_estimation}, we tested the TVAR modeling of respiratory sounds approach after we gave description of AR and TVAR processes. The method in (Koray) uses the basis functions to estimate the TVAR coefficients, we also implemented the windowing based and Kalman filter approaches and measured the performance by looking at the correlation coefficient between estimation and absolute value of airflow. From the results, it can be said that, all three approaches have similar performance. Later, we tested the correlation of magnitudes of different frequency bands with the airflow itself. Lastly, we introduced the Wiener filter approach to unify different estimations of airflow.
\par In chapter \ref{chp:airflow_phase_estimation}, we first gave a method to estimate the period of breathing from the estimation of airflow, which was the first AR coefficient. Then, we worked with a neural network for classification of inspiration and expiration. We presented the histograms of features (AR, Time-Frequency, Percentile Frequencies, Variance, Entropy and Kurtosis) from inspiration and expiration parts. We then presented a method to denoise the output of neural networks by using the period information and assuming a 50\% duty cycle. We also try to estimate the transition points by using the absolute airflow estimation and the period information. 
\par To sum up, the relation between respiratory sounds, their features and the airflow is explored throughout this thesis. It can be concluded that, the phase information can be extracted from respiratory sounds with a good performance by using the techniques in chapter \ref{chp:airflow_phase_estimation} and the airflow curve can be estimated with the techniques in chapter \ref{chp:airflow_curve_estimation} but with a less performance compared to phase estimation.